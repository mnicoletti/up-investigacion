\documentclass[journal]{IEEEtran}

\usepackage[spanish]{babel}
\usepackage[utf8]{inputenc}

\setlength{\parskip}{\baselineskip}

\begin{document}

\title{Impulso Lúdico en el manejo de equipos \textit{DevOps}: Gamificación y Automatización en Acción}

\author{Maximiliano Nicoletti}

\date{2024}

\maketitle

\section{\textbf{\Large Idea}}

El objetivo principal es el de extraer datos de indicadores de performance (\textit{KPI}\footnote{Key Performance Indicators}) de procesos de automatización, integración y despliegue continuo (CI/CD\footnote{Continuous Integration/Continuous Delivery}) en equipos \textit{DevOps}, y utilizarlos para la creación de un sistema de gamificación que permita a las áreas de management obtener información en tiempo real sobre el estado de los proyectos y la performance de los equipos de trabajo, en una plataforma unificada y amigable para la visualización de datos.

Esta plataforma debería permitir:
\begin{itemize}
    \item La visualización de los indicadores de performance en un ámbito lúdico.
    \item La creación de competencias y desafíos entre los equipos de trabajo.
    \item Permitir a los líderes tomar decisiones sobre los próximos pasos en el roadmap de proyectos.
    \item Reducir el gap de conocimiento entre los equipos de desarrollo y operaciones.
    \item Mejorar la toma de decisiones estraéticas en la organización.
\end{itemize}

Para el desarrollo teórico de esta idea, identificaremos los conceptos básicos de gamificación, las prácticas de análisis continuo del feedback en la cultura de trabajo \textit{DevOps} y, posteriormente, la extracción de todos los datos identificados como \textit{KPI} para la creación de un nuevo sistema de visualización gamificado, que permita a los líderes de la organización tomar decisiones.

\section{\textbf{\Large Investigación bibliografica}}
\subsection{\textbf{\large Libros}}

\cite{zichermann2011gamification} \textbf{Gamification by Design} es un libro que provee una guía para el diseño de estrategias para la integración de mecánicas de juego en aplicaciones y servicios de consumo masivo.

\cite{nallar2015estructuraludica} \textbf{Durgan A. Nallar} es co-fundador de la escuela de \textit{Game Design América Latina}, y sus más de 30 años de experiencia en la industria del periodismo del videojuego, la enseñanza del diseño de juegos, y la promoción de este arte, le han dado la perspectiva crítica sobre el diseño de videojuegos que le ha permitido escribir este libro. En el tomo primero de la serie, Nallar dedica un capítulo entero a definir qué es la gamificación y cómo se puede aplicar en distintos ámbitos, ayudando a reconocer elementos clave como métricas, actores y objetivos.

\cite{davis2016effective} \textbf{Effective \textit{DevOps}} es un libro que provee una guía para la implementación de prácticas \textit{DevOps} en organizaciones de desarrollo de software, definiendo los pilares de la cultura \textit{DevOps}, entre los cuales definirá las herramientas y prácticas necesarias para los sistemas de métricas y monitoreo.

\cite{kim2018phoenix} \textbf{The Phoenix Project} es una novela que narra la historia de un gerente de TI que se encuentra en una situación crítica en su empresa, y que deberá aprender a aplicar las prácticas \textit{DevOps} para salvar su trabajo y su empresa. En el libro se describen las prácticas \textit{DevOps} y se presentan los personajes y situaciones que se pueden encontrar en una organización que implementa \textit{DevOps}.

\cite{kim2021devops} \textbf{The \textit{DevOps} Handbook} es otra obra de \textbf{Gene Kim}, en la que se profundiza en las prácticas \textit{DevOps} y se presentan casos de estudio de empresas que han implementado \textit{DevOps} con éxito. Específicamente, \textbf{Kim} describe lo importante de dos aspectos clave en la implementación de \textit{DevOps}: las prácticas del feedback y la importancia de un aprendizaje continuo.

\subsection{\textbf{\large Conferencias}}

\cite{heilbrunn2014towards} Los autores presentan un listado de los requisitos más importantes para la implementación de un sistema de gamificación en un entorno dado. Definen entonces, un análisis teórico de diferentes factores a analizar, como los \textit{KPI} los elementos de juego analíticos, grupos de interés, entre otros.

\cite{meng2014gamification} En este caso, los autores presentan un estudio en el que definen criterios de acercamiento al usuario en la implementación de técnicas de gamificación.

\cite{ayoup2022achievement} Se presenta un caso de estudio donde se implementa un sistema gamificado para la mejora continua de las prácticas \textit{DevOps}. En el caso, se presentan los resultados obtenidos enfocados a la adopción de la cultura \textit{DevOps} desde la aplicación de las nuevas técnicas.

\cite{palenvcarova2022goal} Para el caso de estudio, se realiza una busqueda de diversos \textit{KPI} identificables para su implementación en un entorno gamificado, y la provisión a los equipos de managemente de herramientas esenciales para el análisis de resultados de la operación de equipos de trabajo.

\cite{grande2023serious} Por el contrario de las fuentes de informacíon vistas hasta ahora, se plantea en el caso de estudio el desarrollo e implementación de un videojuego para la adopción de la cultura \textit{DevOps} desde un punto de vista ameno y lúdico del aprendizaje continuo.

\subsection{\textbf{\large Artículos}}

\cite{brunnert2015performance} En el artículo, los autores analizan la importancia de la medición de indicadores de performance en la aplicación de las técnicas de la cultura \textit{DevOps} de automatización, integración y despliegue continuo. De esta forma, buscan identificar la correlación entre el análisis de estas métricas, y la definición de actividades para la iteración veloz entre los equipos de desarrollo y operaciones.

\cite{amaro2024devops} En el artículo del \textbf{Instituto Universitorio de Lisboa}, los autores tienen como objetivo la identificación de las métricas y \textit{KPIs} más importantes de las áreas de desarrollo y operaciones, y las formas de extracción, transformación y muestra de las mismas para la toma de decisiones.

\subsection{\textbf{\large Tesis Universitarias}}

\cite{kruis2014designing} En su trabajo de tesis para el título de Magister en la \textbf{Universidad de Tecnología de Eindhoven}, el autor nos provee con un nuevo modelo de análisis de métricas dedicadas a la cultura \textit{DevOps}, su extracción, análisis y diseño de una solución de visualización para afectar la obtención de resultados en la toma de decisiones.

\cite{heilbrunn2014towards} En su tesis doctoral, \textbf{Benjamin Helbrunn} describe un extensivo modelo de análisis de métricas mediante la implementación de un sistema detallado de gamificación en gran detalle técnico.

\cite{mueller2024leveraging} Por último, \textbf{Stefan Mueller} investiga el potencial de la integración de técnicas de gamificación en las operaciones diarias de equipos \textit{DevOps}, para buscar la motivación de los equipos de trabajo y la mejora continua.

\bibliographystyle{IEEEtran}
\bibliography{BibliografiaGamification}

\end{document}