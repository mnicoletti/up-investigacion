\documentclass[journal]{IEEEtran}

\usepackage[spanish]{babel}
\usepackage[utf8]{inputenc}
\usepackage{tcolorbox}
\usepackage{xcolor}
\usepackage{hyperref}

\setlength{\parskip}{\baselineskip}

\begin{document}

\title{Impulso Lúdico en el manejo de equipos \textit{DevOps}: Gamificación y Automatización en Acción}

\author{Maximiliano Nicoletti}

\date{2024}

\maketitle

% Idea de la investigación: Esta sección será retirada en la versión final del trabajo, al incluir Abstract, Introducción y Conclusiones.
\section{\textbf{\Large Idea}}

El objetivo principal es el de extraer datos de indicadores de performance (\textit{KPI}\footnote{Key Performance Indicators}) de procesos de automatización, integración y despliegue continuo (CI/CD\footnote{Continuous Integration/Continuous Delivery}) en equipos \textit{DevOps}, y utilizarlos para la creación de un sistema de gamificación que permita a las áreas de \textit{management} obtener información en tiempo real sobre el estado de los proyectos y la performance de los equipos de trabajo, en una plataforma unificada y amigable para la visualización de datos.

Esta plataforma debería permitir:
\begin{itemize}
    \item La visualización de los indicadores de performance en un ámbito lúdico.
    \item La creación de competencias y desafíos entre los equipos de trabajo.
    \item Permitir a los líderes tomar decisiones sobre los próximos pasos en el \textit{roadmap} de proyectos.
    \item Reducir el \textit{gap} de conocimiento entre los equipos de desarrollo y operaciones.
    \item Mejorar la toma de decisiones estratégicas en la organización.
\end{itemize}

Para el desarrollo teórico de esta idea, identificaremos los conceptos básicos de gamificación, las prácticas de análisis continuo del \textit{feedback} en la cultura de trabajo \textit{DevOps} y, posteriormente, la extracción de todos los datos identificados como \textit{KPI} para la creación de un nuevo sistema de visualización gamificado, que permita a los líderes de la organización tomar decisiones.

% Sección 1: Marco Teórico
\section{\textbf{\Large Marco Teórico}}

En esta sección identificaremos los conceptos que abarcarán el total de la investigación, desde los pilares de la cultura \textit{DevOps}, identificando la importancia de una correcta visualización de los indicadores de performance, y el concepto de gamificación como herramienta de visualización. La revisión de estos conceptos permitirá definir la base teórica para la implementación de la plataforma propuesta.

% Sección 1.1: Fundamentos de DevOps y Automatización
\subsection{\textbf{Fundamentos de DevOps y Automatización}}

\begin{tcolorbox}[colback=gray!10, colframe=black, left=2mm, right=2mm]
    \small % Reduce el tamaño de la fuente ligeramente
    \ttfamily % Aplica la fuente monoespaciada
    \raggedright % Alinea el texto a la izquierda sin justificar
    Mientras que las bases de \textit{DevOps} se pueden ver como derivados de \textbf{LEAN}, \textbf{la teoría de las restricciones} y el movimiento \textbf{Toyota Kata}, muchos también ven a \textit{DevOps} como una continuación lógica al camino del software Agile que inició en el año 2001 \cite{kim2021devops}.
    \end{tcolorbox}

La cultura \textit{DevOps} está basada en la colaboración entre todos los actores de los equipos de desarrollo y operaciones, mejorando la productividad y la calidad de los productos. Y, si bien hoy podemos estar acostumbrados a describir \textit{DevOps} como un conjunto de herramientas con nombre propio que se utilizan para la automatización de procesos, a la vez que no existe una definición única de \textit{DevOps}.
Lo que podemos afirmar sobre \textit{DevOps} es que utiliza una combinación de cambios culturales y estrategias tecnológicas para llevar a cabo los objetivos en los que se basa, mencionados anteriormente.
De acuerdo con Gene Kim et al. en sus libros \textit{The DevOps Handbook} \cite{kim2021devops} y \textit{The Phoenix Project} \cite{kim2018phoenix}, es posible identificar tres principios bajo los que enmarcar todos los patrones y conductas que se observan en la adopción de \textit{DevOps}:
\begin{enumerate}
    \item \textbf{El principio del flujo}\cite[Part 1.2: The First Way: The Principles of Flow]{kim2021devops}: En este primer principio, se busca disminuir el tiempo en la puesta en producción de un cambio, para lo que buscamos visualizar el trabajo que se está realizando, limitar el \textbf{WIP}\footnote{Work In Process (Trabajo en Proceso)} y reducir el tamaño de "lotes" de trabajo.
    \item \textbf{El principio del feedback}\cite[Part 1.3: The Second Way: The Principles of Feedback]{kim2021devops}: Al trabajar con sistemas complejos, cobra especial importancia una retroalimentación rápida y efectiva, aumentando la cantidad y calidad de información que permita identificar y hasta predecir posibles fallas. Este segundo principio busca atacar los problemas al mismo tiempo que están ocurriendo para construir conocimiento constantemente y optimizar cada proceso.
    \item \textbf{El principio de la experimentación y aprendizaje continuo}\cite[Part 1.4: The Third Way: The Principles of Continual Learning and Experimentation]{kim2021devops}: En el tercer principio, buscamos crear una cultura donde la experimentación y el aprendizaje continuo sean la norma. Entre muchos otros formatos, este último principio refleja específicamente el papel que debe tener un líder, no solo como hacedor de objetivos y roadmaps, sino también como facilitador del aprendizaje continuo.
\end{enumerate}

% Al avanzar en el paper es posible que esta sección desaparezca y termine en la introducción.
Si bien cada principio de esta breve definición de la cultura \textit{DevOps} funciona de forma autónoma, es fácilmente observable el papel de retroalimentación que juega cada uno de ellos en el resto, y es sobre esta serie de principios que plantearemos los objetivos de nuestra solución: la obtención de datos de indicadores dentro de nuestro flujo de trabajo, el constante feedback de estos datos a nuestra plataforma, y la visualización de estos datos para el beneficio conjunto de los equipos de trabajo y los líderes de la organización.
% Sección 1.2: Principios de Gamificación en Entornos Empresariales
\subsection{\textbf{Principios de Gamificación en Entornos Empresariales}}

\textit{Desarrollo esperado:} Se definirán los elementos de gamificación y su aplicabilidad en ambientes de trabajo, incluyendo la influencia de sistemas de juego como los \textbf{dados de 20 caras} en la promoción de la innovación.

% Sección 1.3: Importancia de los KPIs en DevOps
\subsection{\textbf{Importancia de los KPIs en DevOps}}

\textit{Desarrollo esperado:} Se explicará qué son los \textit{KPIs} y su papel en la medición del rendimiento en equipos \textit{DevOps}. Se resaltará que \textbf{``Un dato aislado no provee contexto''}, subrayando la necesidad de analizar múltiples indicadores para obtener una visión completa.

Ejemplos de \textit{KPIs} relevantes incluyen:

\begin{itemize}
    \item \textbf{Frecuencia de Despliegues:}
    
    \textit{Descripción:} Indica cuántas veces se realizan despliegues a producción en un periodo determinado.
    
    \textit{Relevancia:} Una alta frecuencia sugiere que el equipo tiene procesos ágiles y automatizados, reduciendo el \textit{gap} entre desarrollo y operaciones.
    
    \textit{Ejemplo:} Un equipo que despliega diariamente muestra una cultura \textit{DevOps} madura, mientras que uno que lo hace mensualmente podría enfrentar retrasos y acumulación de cambios.
    
    \item \textbf{Tiempo Medio de Recuperación (MTTR):}
    
    \textit{Descripción:} Tiempo promedio que se tarda en restaurar un servicio después de una interrupción.
    
    \textit{Relevancia:} Un MTTR bajo refleja una eficiente colaboración en la resolución de problemas y robustez en los procesos.
    
    \textit{Ejemplo:} Reducir el MTTR de horas a minutos puede mejorar significativamente la disponibilidad del servicio y la satisfacción del cliente.
    
    \item \textbf{Disponibilidad del Servicio (\textit{Service Uptime}):}
    
    \textit{Descripción:} Porcentaje de tiempo que un servicio está operativo y disponible para los usuarios.
    
    \textit{Relevancia:} Una alta disponibilidad es esencial para la confianza del cliente y refleja una gestión efectiva de operaciones.
    
    \textit{Ejemplo:} Mantener una disponibilidad del 99.9\% demuestra compromiso con la calidad y eficiencia en la gestión de infraestructuras.
\end{itemize}

% Sección 2: Situación Actual y Desafíos en Equipos DevOps
\section{\textbf{\Large Situación Actual y Desafíos en Equipos DevOps}}

\textbf{Resumen:} Se analizará la situación actual de los equipos \textit{DevOps}, identificando las complicaciones existentes en la interpretación de datos y en la colaboración entre equipos.

% Sección 2.1: Importancia de los KPIs y Desafíos en su Interpretación
\subsection{\textbf{Importancia de los KPIs y Desafíos en su Interpretación}}

\textit{Desarrollo esperado:} Se analizará cómo los \textit{KPIs} son esenciales para medir el rendimiento y se discutirán las dificultades que enfrentan los equipos al interpretar datos aislados, reforzando la idea de que \textbf{``Un dato aislado no provee contexto''}. Se explorará cómo la falta de contexto puede llevar a decisiones erróneas.

% Sección 2.2: Brecha entre Equipos de Desarrollo y Operaciones
\subsection{\textbf{Brecha entre Equipos de Desarrollo y Operaciones}}

\textit{Desarrollo esperado:} Se explorará el \textit{gap} de conocimiento entre los equipos, y cómo afecta la eficiencia y la toma de decisiones estratégicas. Se destacará la necesidad de una comunicación efectiva y colaboración para superar esta brecha, enfatizando además en la necesidad de que los mandos intermedios actúen como facilitadores.

Un caso de estudio relevante es:

\begin{itemize}
    \item \textbf{Retrasos en el Despliegue por Procesos Manuales}
    
    \textit{Descripción:} El equipo de desarrollo completa nuevas funcionalidades, pero el equipo de operaciones tiene procedimientos manuales y lentos para el despliegue.
    
    \textit{Impacto:} El tiempo de comercialización se incrementa, dando ventaja a competidores más ágiles.
    
    \textit{Relevancia:} Destaca la importancia de la automatización y procesos CI/CD integrados para reducir el \textit{gap}.
\end{itemize}

% Sección 3: Gamificación como Herramienta de Mejora
\section{\textbf{\Large Gamificación como Herramienta de Mejora}}

\textbf{Resumen:} Esta sección explora la pregunta de cómo la gamificación puede resolver las complicaciones identificadas en la sección anterior. Se presentará la gamificación como respuesta, destacando cómo promueve la competitividad y la innovación.

% Sección 3.1: Integración de Gamificación en Equipos de Trabajo
\subsection{\textbf{Integración de Gamificación en Equipos de Trabajo}}

\textit{Desarrollo esperado:} Se analizará cómo la gamificación puede ser aplicada en entornos \textit{DevOps} para mejorar el compromiso y la colaboración. Se enfatizará que \textbf{``La competitividad es promotora de la innovación''}, mostrando ejemplos de cómo los desafíos y competencias pueden impulsar el rendimiento.

% Sección 3.2: Diseño Conceptual de la Plataforma
\subsection{\textbf{Diseño Conceptual de la Plataforma}}

\textit{Desarrollo esperado:} Se describirá cómo la plataforma integrará la visualización de \textit{KPIs} en un entorno lúdico, facilitando la comprensión y promoviendo la competitividad.

Elementos de gamificación considerados:

\begin{itemize}
    \item \textbf{Competencias y Clasificaciones (\textit{Leaderboards}):}
    
    \textit{Elementos de Gamificación:} Tablas de clasificación que muestran el rendimiento de los equipos en tiempo real.
    
    \textit{Transformación de \textit{KPIs}:} Los equipos se posicionan según su tasa de éxito en despliegues o tiempo de ciclo.
    
    \textit{Beneficio:} Fomenta una competencia saludable que impulsa la innovación y mejora del rendimiento.
\end{itemize}

% Sección 4: Implementación y Funcionamiento de la Plataforma
\section{\textbf{\Large Implementación y Funcionamiento de la Plataforma}}

\textbf{Resumen:} Se explicará en detalle cómo se extraerán y utilizarán los datos de \textit{KPIs}, y cómo funcionará la plataforma en la práctica.

% Sección 4.1: Extracción y Análisis de Datos de KPIs
\subsection{\textbf{Extracción y Análisis de Datos de KPIs}}

\textit{Desarrollo esperado:} Se describirá el proceso de extracción de datos de los procesos CI/CD y cómo estos pueden ser transformados en indicadores útiles. Se abordará la necesidad de automatización en este proceso para asegurar datos en tiempo real.

% Sección 4.2: Interfaz de Usuario y Experiencia Lúdica
\subsection{\textbf{Interfaz de Usuario y Experiencia Lúdica}}

\textit{Desarrollo esperado:} Se describirá la interfaz de la plataforma, haciendo referencia a elementos como los \textbf{juegos de dados de 20 caras} para hacer la experiencia más atractiva. Se propondrán interfaces interactivas y elementos de juego que faciliten la comprensión y el seguimiento del rendimiento.

% Sección 4.3: Herramientas para Líderes y Mandos Medios
\subsection{\textbf{Herramientas para Líderes y Mandos Medios}}

\textit{Desarrollo esperado:} Se presentarán las funcionalidades que permitirán a los líderes tomar decisiones estratégicas basadas en la información proporcionada.

Aplicación en la plataforma:

\begin{itemize}
    \item \textbf{Sistemas de Seguimiento de Objetivos (OKRs):}
    
    \textit{Experiencia Real:} Implementación de OKRs para alinear esfuerzos y medir resultados.
    
    \textit{Aplicación en la Plataforma:} Permitir que los líderes definan OKRs y que el progreso hacia ellos se refleje en el entorno gamificado.
    
    \textit{Beneficio:} Alinea las actividades diarias con la estrategia general de la organización.
\end{itemize}

% Sección 5: Impacto en la Organización y Beneficios Esperados
\section{\textbf{\Large Impacto en la Organización y Beneficios Esperados}}

\textbf{Resumen:} Esta sección analiza las implicaciones de implementar la plataforma propuesta, incluyendo los beneficios para la organización y posibles desafíos a considerar.

% Sección 5.1: Fomento de una Cultura de Innovación
\subsection{\textbf{Fomento de una Cultura de Innovación}}

\textit{Desarrollo esperado:} Se destacará cómo la gamificación y la competencia saludable pueden promover una cultura de innovación continua dentro de los equipos \textit{DevOps}.

% Sección 5.2: Alineación con Objetivos Estratégicos
\subsection{\textbf{Alineación con Objetivos Estratégicos}}

\textit{Desarrollo esperado:} Se explicará cómo la información proporcionada por la plataforma ayuda a los líderes a alinear las acciones de los equipos con los objetivos estratégicos de la organización.

% Sección 5.3: Reducción del Gap entre Equipos
\subsection{\textbf{Reducción del \textit{Gap} entre Equipos}}

\textit{Desarrollo esperado:} Se analizará cómo la plataforma ayuda a reducir la brecha de conocimiento entre desarrollo y operaciones, y entre niveles de experiencia, proveyendo al liderazgo de una visión más completa y facilitando la provisión de herramientas de apoyo en \textit{onboarding} y capacitación.

% Sección 6: Retos y Consideraciones para la Implementación
\section{\textbf{\Large Retos y Consideraciones para la Implementación}}

\textbf{Resumen:} Se identificarán posibles desafíos en la implementación y se propondrán soluciones.

% Sección 6.1: Resistencia al Cambio y Cultura Organizacional
\subsection{\textbf{ Resistencia al Cambio y Cultura Organizacional}}

\textit{Desarrollo esperado:} Se abordarán las posibles resistencias internas y cómo superarlas.

Ejemplo:

\begin{itemize}
    \item \textbf{Resistencia por Rutinas Arraigadas}
    
    \textit{Descripción:} Preferencia por mantener métodos tradicionales conocidos.
    
    \textit{Superación:} Involucrar a los empleados en el proceso de cambio, escuchando sus preocupaciones y ajustando el plan según sea necesario.
\end{itemize}

% Sección 6.2: Implicaciones Éticas y de Privacidad
\subsection{\textbf{ Implicaciones Éticas y de Privacidad}}

\textit{Desarrollo esperado:} Se analizarán consideraciones éticas relacionadas con la competencia entre equipos y el manejo de datos sensibles, asegurando que la plataforma promueva un entorno de trabajo positivo.

% Sección 6.3: Escalabilidad y Mantenimiento de la Plataforma
\subsection{\textbf{ Escalabilidad y Mantenimiento de la Plataforma}}

\textit{Desarrollo esperado:} Se analizarán los aspectos técnicos para garantizar el funcionamiento continuo y la adaptación futura de la plataforma.

Herramientas de código abierto como base:

\begin{itemize}
    \item \textbf{Integración con Prometheus y Grafana}
    
    \textit{Descripción:} Utilizar Prometheus para recopilar métricas y Grafana para visualizarlas.
    
    \textit{Aplicación:} Crear \textit{dashboards} gamificados sobre Grafana, incorporando \textit{plugins} o desarrollos que añadan elementos lúdicos.
\end{itemize}

% Sección de Citas. Esta sección se completará con la investigación bibliográfica una vez finalizado el paper, y se incluirá en la versión final del trabajo.
\section{\textbf{\Large \textbf{\Large Investigación bibliografica}}}
\subsection{\textbf{ \textbf{\large Libros}}}

\cite{zichermann2011gamification} \textbf{Gamification by Design} es un libro que provee una guía para el diseño de estrategias para la integración de mecánicas de juego en aplicaciones y servicios de consumo masivo.

\cite{nallar2015estructuraludica} \textbf{Durgan A. Nallar} es co-fundador de la escuela de \textit{Game Design América Latina}, y sus más de 30 años de experiencia en la industria del periodismo del videojuego, la enseñanza del diseño de juegos, y la promoción de este arte, le han dado la perspectiva crítica sobre el diseño de videojuegos que le ha permitido escribir este libro. En el tomo primero de la serie, Nallar dedica un capítulo entero a definir qué es la gamificación y cómo se puede aplicar en distintos ámbitos, ayudando a reconocer elementos clave como métricas, actores y objetivos.

\cite{davis2016effective} \textbf{Effective \textit{DevOps}} es un libro que provee una guía para la implementación de prácticas \textit{DevOps} en organizaciones de desarrollo de software, definiendo los pilares de la cultura \textit{DevOps}, entre los cuales definirá las herramientas y prácticas necesarias para los sistemas de métricas y monitoreo.

\cite{kim2018phoenix} \textbf{The Phoenix Project} es una novela que narra la historia de un gerente de TI que se encuentra en una situación crítica en su empresa, y que deberá aprender a aplicar las prácticas \textit{DevOps} para salvar su trabajo y su empresa. En el libro se describen las prácticas \textit{DevOps} y se presentan los personajes y situaciones que se pueden encontrar en una organización que implementa \textit{DevOps}.

\cite{kim2021devops} \textbf{The \textit{DevOps} Handbook} es otra obra de \textbf{Gene Kim}, en la que se profundiza en las prácticas \textit{DevOps} y se presentan casos de estudio de empresas que han implementado \textit{DevOps} con éxito. Específicamente, \textbf{Kim} describe lo importante de dos aspectos clave en la implementación de \textit{DevOps}: las prácticas del feedback y la importancia de un aprendizaje continuo.

\subsection{\textbf{ \textbf{\large Conferencias}}}

\cite{heilbrunn2014towards} Los autores presentan un listado de los requisitos más importantes para la implementación de un sistema de gamificación en un entorno dado. Definen entonces, un análisis teórico de diferentes factores a analizar, como los \textit{KPI} los elementos de juego analíticos, grupos de interés, entre otros.

\cite{meng2014gamification} En este caso, los autores presentan un estudio en el que definen criterios de acercamiento al usuario en la implementación de técnicas de gamificación.

\cite{ayoup2022achievement} Se presenta un caso de estudio donde se implementa un sistema gamificado para la mejora continua de las prácticas \textit{DevOps}. En el caso, se presentan los resultados obtenidos enfocados a la adopción de la cultura \textit{DevOps} desde la aplicación de las nuevas técnicas.

\cite{palenvcarova2022goal} Para el caso de estudio, se realiza una busqueda de diversos \textit{KPI} identificables para su implementación en un entorno gamificado, y la provisión a los equipos de managemente de herramientas esenciales para el análisis de resultados de la operación de equipos de trabajo.

\cite{grande2023serious} Por el contrario de las fuentes de informacíon vistas hasta ahora, se plantea en el caso de estudio el desarrollo e implementación de un videojuego para la adopción de la cultura \textit{DevOps} desde un punto de vista ameno y lúdico del aprendizaje continuo.

\subsection{\textbf{ \textbf{\large Artículos}}}

\cite{brunnert2015performance} En el artículo, los autores analizan la importancia de la medición de indicadores de performance en la aplicación de las técnicas de la cultura \textit{DevOps} de automatización, integración y despliegue continuo. De esta forma, buscan identificar la correlación entre el análisis de estas métricas, y la definición de actividades para la iteración veloz entre los equipos de desarrollo y operaciones.

\cite{amaro2024devops} En el artículo del \textbf{Instituto Universitorio de Lisboa}, los autores tienen como objetivo la identificación de las métricas y \textit{KPIs} más importantes de las áreas de desarrollo y operaciones, y las formas de extracción, transformación y muestra de las mismas para la toma de decisiones.

\subsection{\textbf{ \textbf{\large Tesis Universitarias}}}

\cite{kruis2014designing} En su trabajo de tesis para el título de Magister en la \textbf{Universidad de Tecnología de Eindhoven}, el autor nos provee con un nuevo modelo de análisis de métricas dedicadas a la cultura \textit{DevOps}, su extracción, análisis y diseño de una solución de visualización para afectar la obtención de resultados en la toma de decisiones.

\cite{heilbrunn2014towards} En su tesis doctoral, \textbf{Benjamin Helbrunn} describe un extensivo modelo de análisis de métricas mediante la implementación de un sistema detallado de gamificación en gran detalle técnico.

\cite{mueller2024leveraging} Por último, \textbf{Stefan Mueller} investiga el potencial de la integración de técnicas de gamificación en las operaciones diarias de equipos \textit{DevOps}, para buscar la motivación de los equipos de trabajo y la mejora continua.

\bibliographystyle{IEEEtran}
\bibliography{BibliografiaGamification}

\end{document}